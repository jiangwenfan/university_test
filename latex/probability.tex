%导言区
\documentclass{article}
\title{概率论与数理统计}
\author{jason stark}
\date{\today}

\usepackage{ctex}


%conetnt
\begin{document}
	\maketitle
	
	1.引言.
	确定性(必然事件):一定发生 或 一定不发生
	随机性(偶尔): 可能发生 或 可能不发生
	统计规律
	
	1.1随机事件
	随机试验:用E表示
	事件:每种结果
	随机事件:可能发生,可能不发生 
	基本事件:不能再分(不必再分)
	符合事件:由基本事件复合
	
	必然事件-->$\Omega$-->样本空间
	不可能事件-->$\emptyset$-->空集
	
	2.并集(和)
	A $\bigcup$ B , A+B , A与B至少有一个发生。
	
	3.交集(积)
	A $\bigcap B, AB, A、B同时发生
	
	4.A-B 
	A发生B不发生,属于A而不属于B
	
	5.互不相容事件
	A,B不同时发生,AB=$\emptyset$
	
	6.对立事件
	A,B互不相容,且A$\bigcup$B=$\Omega$ , AB=$\emptyset$ ,且A+B=$\Omega$
	A=$\bar{B}$, B=$\bar{A}$
	
	A是$\bar{A}$的逆,所以$\bar{\bar{A}}$=A 
	A-B=A-AB=A$\bar{B}$
	
	互不相容适用于多个事件,对立事件只用于2个事件。
	互不相容不能同时发生,可以都不发生。
	对立事件,又且只有一个事件发生。
	
	完备事件:
	两两互不相容。
	
	运算律:
	1).交换律
	2).结合律
	3).分配律
	4).对偶律
	
	2.古典概率模型
	2.1条件:
		有限个样本点
		等可能性
	排列组合:
		是A或B, A+B种。
		先A后B, A$\times$B种。
		1).不放回。从n个不同元素中取出m个。不重复排列
			$$C_n^m=n(n-1)(n-2)\cdots(n-m+1)=\frac{n!}{(n-m)!}$$
		2).全排列
			$$C_n^n=n(n-1)(n-2)\cdots\times2\times1$$
		重复排列:放回
		从n中取m个
		$$n \times n \times n \cdots \times n = n^n$$
		
		组合:
		从n个不同的元素中取出m个不同的元素。
		$$C_n^m = \frac{n!}{m!(n-m)!}$$
		$$C_n^m = C_n^{n-m}$$
		$$C_n^0 = C_n^n = 1$$
		
		例1.有一套五本选集的书,从左-->右或从右-->左是1,2,3,4,5的概率是多少?
		$$\frac{2}{C_5^5} = \frac{1}{60}$$
		例2.有4个邮筒,2封信。
		1).前两个邮筒给投一封信的概率。
		$$\frac{C_2^2}{4\times4} = \frac{1}{4}$$
		2).第2个邮筒有一个信的概率。
		$$\frac{C_2^1 C_3^1}{4\times4} = \frac{3}{8}$$
		例3.有5个白球,4个黑球,任取3个球。
		1).取出2白1黑的概率
		$$\frac{C_5^2 C_4^1}{C_9^3}$$
		2).取出没黑球的概率
		$$\frac{C_5^3}{C_9^3}$$
		3).取出颜色相同的概率
		$$\frac{C_5^3 + C_4^3}{C_9^3}$$
	3.几何概型
	4.频率与概率
	5.公理化
	6.条件概率
	7.乘法公式
	8.全概率公式
	9.贝叶斯公式
	10.事件的独立性
	11.伯努利模型
	12.随机变量
	12.1.离散型随机变量及其概率分布
	12.2.连续性随机变量及其密度函数
	13.分布函数的定义
	13.1离散型分布函数
	13.2连续型分布函数
	13.3 0-1分布
	13.4 几何分布
	13.5 二项分布
	13.6 泊松分布
	13.7 超几何分布
	13.8 均匀分布
	13.9 指数分布
	13.10 正态分布
	
	
	
		
\end{document}
